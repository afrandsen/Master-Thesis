\begin{titlingpage}
\begin{center}
\vspace*{0.2cm}

\LARGE \textbf{LANGSIGTET INVESTERING I AKTIVER}

\vspace{3mm}

\renewcommand{\thefootnote}{\fnsymbol{footnote}}
\large \textbf{EN EMPIRISK ANALYSE AF PORTEFØLJEALLOKERING MED REBALANCERING}

\tiny LONG-TERM ASSET INVESTMENT\\ AN EMPIRICAL ANALYSIS OF PORTFOLIO ALLOCATION WITH REBALANCING

\vspace{2mm}

\Large Andreas Kracht Frandsen\footnote{Institut for Matematik, Aarhus Universitet, andreas.kracht.frandsen@post.au.dk.}

\Large 201506176

\Large Vejleder: Prof. Jan Pedersen\footnote{Institut for Matematik, Aarhus Universitet, jan@math.au.dk.}

\vspace{2mm}

\rule{1cm}{0.4pt}

\vspace{2mm}

Speciale i Matematik-Økonomi\\
Juni 2020

\vfill

\includegraphics[width=0.4\textwidth]{latex/ausegl_sort}

\vfill

\textbf{Abstract}
\end{center}
\begin{center}
\begin{minipage}{12cm}
    Empirical research in finance has shown that expected excess returns and risk on assets shift over time in predictable ways this affects the long-horizon investor, in terms of the risk-return tradeoff. Further, this makes the traditional -- Markowitz -- portfolio theory invalid for the long-horizon investor. In this paper I describe the dynamics of returns using an empirical model, that is based on a vector autoregressive process of lag order one. I analyze the data to get an understanding of how risk changes over time. In addition, I use the results obtained to examine efficient asset allocation over investment horizons. I illustrate the approach using data from \textit{NYSE}, \textit{AMEX}, \textit{NASDAQ} and the U.S bonds market. I find that asset return predictability has significant effects on the variance and correlation structure of returns on stocks, bonds and T-Bills across investment horizons. Further, I find that the efficient portfolios over long horizons differ relatively much from the one-period efficient portfolios.Empirical research in finance has shown that expected excess returns and risk on assets shift over time in predictable ways this affects the long-horizon investor, in terms of the risk-return tradeoff. Further, this makes the traditional -- Markowitz -- portfolio theory invalid for the long-horizon investor. In this paper I describ
\end{minipage}
\end{center}
\end{titlingpage}

\newpage

\vspace*{\fill}

Copyright \textcopyright\, 2020 \TeX nician og useR Andreas Kracht Frandsen

Alt indhold er licenseret under CC BY-NC-SA 4.0

\textit{Sat med Palatino Linotype}

\newpage
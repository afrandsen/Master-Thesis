\begin{titlingpage}
\begin{center}
\LARGE \textbf{LANGSIGTET INVESTERING I AKTIVER}

\vspace{2mm}

\renewcommand{\thefootnote}{\fnsymbol{footnote}}
\large \textbf{EN EMPIRISK ANALYSE AF PORTEFØLJEALLOKERING MED REBALANCERING}

\tiny LONG-TERM ASSET INVESTMENT\\ AN EMPIRICAL ANALYSIS OF PORTFOLIO ALLOCATION WITH REBALANCING

\vspace{2mm}

\Large Andreas Kracht Frandsen\footnote{Kreditmodeller, Jyske Bank, andreas.kracht.frandsen@jyskebank.dk}

\Large 201506176

\Large Vejleder: Jan Pedersen\footnote{Institut for Matematik, Aarhus Universitet, jan@math.au.dk.}

\vspace{2mm}

\rule{1cm}{0.4pt}

\vspace{2mm}

Speciale i Matematik-Økonomi\\
Juni 2020

\vfill

\includegraphics[width=0.35\textwidth]{latex/ausegl_sort.pdf}

\vfill

\end{center}
\begin{center}
\textbf{Abstract}

\begin{minipage}{16cm}
Empirical research in finance has shown that expected excess returns and risk
on assets shift over time in predictable ways. This affects the long-horizon investor, in terms of the risk-return tradeoff. Further, this makes the traditional -– Markowitz –- portfolio theory invalid for the long-horizon investor. This thesis explore the finance field of dynamic asset allocation through an institutional investors' perspective using quarterly data spanning from the first quarter of 1954 to the fourth quarter of 2018. To do so, analytical solutions for the institutional investor with power utility defined over their respective budget constraint are utilized. Further the institutional investor is assumed to experience time-varying investment opportunities over a finite investment horizon. In this way, various analyses can be carried out, to study institutional investors' with a short and long-term horizon with various risk tolerance. To make a relevant comparison, all the institutional investors' face the same investment oppurtunity set. The investment universe consist of a benchmark asset 90-day Treasury Bill, an index that includes all stocks traded in the NYSE, AMEX, and NASDAQ markets, the 10-year U.S. Government Bond, and a corporate bond index. Further the most studied predicting variables from existing literature is used to explain variation in excess returns, by examining financial variables, term structure variables and one macroeconomic variable. This examination is followed by a dynamic return model, where the state variables are well described by a first-order vector autoregressive process, and finds that the book-to-market ratio, small-minus-big, and the yield spread are the most comprehensive state variables to do so. This dynamic return model allows one to assess the differences between the institutional investors' myopic and intertemporal heding demand while examining risk tolerance, horizon effects, and how short-sales affect the total portfolio allocation strategy. The study find that stocks are being driven by the intertemporal hedging demand, while the myopic demand drives corporate and government bonds. Allowing for short-sales, the risk tolerant investor retains a short position in Treasury Bills at all horizons. Short-sale constraints make a significant difference to the relative asset allocation decision.
\end{minipage}
\end{center}
\end{titlingpage}

\newpage

\vspace*{\fill}

Copyright \textcopyright\, 2020 \TeX nician og useR Andreas Kracht Frandsen

Alt indhold er licenseret under CC BY-NC-SA 4.0

\textit{Sat med Palatino Linotype}

\includegraphics[width=0.15\linewidth]{latex/bync}
\newpage